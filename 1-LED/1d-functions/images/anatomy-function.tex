\documentclass[xcolor=table]{article}
\usepackage{ragged2e}
\usepackage{pstricks}
\usepackage{pst-eps}
\usepackage{pstricks-add}
\usepackage{anyfontsize}
\usepackage{pifont}
\usepackage{DejaVuSansMono}
\usepackage{libertine}
\begin{document}
\TeXtoEPS
\begin{pspicture}(0,0)(30,20)
\fontsize{20}{22}\selectfont
%\rput[bl](0,0){\psgrid(0,0)(30,20)}
\rput[bl](0,0){%
\begin{minipage}[t]{2.0\linewidth}
\fontfamily{DejaVuSansMono-TLF}\selectfont%
void loop \textcolor{red}{ (}void\textcolor{red}{) \{} \\[10pt]
\hspace*{1cm}\textcolor{black}{digitalWrite(rLED, HIGH);} \\[10pt]
\hspace*{1cm}\textcolor{black}{delay(2000);} \\[10pt]
\hspace*{1cm}\textcolor{black}{digitalWrite(rLED, LOW);} \\[10pt]
\hspace*{1cm}\textcolor{black}{digitalWrite(yLED, HIGH);} \\[10pt]
\hspace*{1cm}\textcolor{black}{delay(250);} \\[10pt]
\hspace*{1cm}\textcolor{black}{digitalWrite(yLED, LOW);} \\[10pt]
\hspace*{1cm}\textcolor{black}{digitalWrite(yLED, HIGH);} \\[10pt]
\hspace*{1cm}\textcolor{black}{delay(1000);} \\[10pt]
\hspace*{1cm}\textcolor{black}{digitalWrite(yLED, LOW);} \\[10pt]
\hspace*{1cm}\textcolor{gray}{return(void);} \\
\textcolor{red}{\}}\\
\end{minipage}
}

\libertine%
\fontshape{it}\fontsize{30}{36}\selectfont%
%
% Return type
%
\rput[b](0.9,17.5){\textcolor{blue}{return type}}
\rput[t](0.9,17.9){\psbrace[linecolor=blue,braceWidthInner=10pt,braceWidthOuter=20pt,linewidth=0.04](0.8,-4.4)(-0.9,-4.4){}}
\rput[t](0.9,17.7){\psline[linewidth=0.05,linecolor=blue](0,-3.1)(0,0)}
%
% Function name
%
\rput[b](3.1,16.2){\textcolor{blue}{name}}
\rput[t](3.1,16.4){\psbrace[linecolor=blue,braceWidthInner=10pt,braceWidthOuter=20pt,linewidth=0.04](0.9,-2.9)(-0.9,-2.9){}}
\rput[t](3.1,16.2){\psline[linewidth=0.05,linecolor=blue](0,-0.3)(0,-1.7)}
%
% Parameters
%
\rput[b](6.35,14.6){\textcolor{blue}{parameter(s)}}
\rput[t](6.35,14.7){\psbrace[linecolor=blue,braceWidthInner=10pt,braceWidthOuter=20pt,linewidth=0.04](0.25,-1.2)(-1.85,-1.2){}}
%\rput[t](6.35,20.8){\psline[linewidth=0.05,linecolor=blue](0,-0.5)(0,-1.0)}
%
% Statements
%
\rput[l](13.8,7.0){\parbox[l]{11in}{\textcolor{blue}{\raggedright statement(s) wrapped in braces}}}
\rput[l](13.4,7.2){\psbrace[linecolor=blue,braceWidthInner=20pt,braceWidthOuter=20pt,linewidth=0.04](-1.4,-6.0)(-1.4,5.8){}}
%
% return statement
%
\rput[l](15.0,1.8){\textcolor{blue}{\textup{return} statement}}
\rput[l](14.8,1.8){\psbrace[linecolor=blue,braceWidthInner=5pt,braceWidthOuter=5pt,linewidth=0.04](-0.9,-0.5)(-0.9,0.5){}}
\end{pspicture}
\endTeXtoEPS
\end{document}
